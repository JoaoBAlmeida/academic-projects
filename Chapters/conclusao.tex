\chapter{Conclusão}
\label{chap:conc}

Chegou a hora de apresentar o apanhado geral sobre o trabalho de
pesquisa feito, no qual s\~ao sintetizadas uma s\'erie de
reflex\~oes sobre a metodologia usada, sobre os achados e
resultados obtidos, sobre a confirma\c{c}\~ao ou recha\c{c}o da
hip\'otese estabelecida e sobre outros aspectos da pesquisa que
s\~ao importantes para validar o trabalho. Recomenda-se n\~ao
citar outros autores, pois a conclus\~ao \'e do pesquisador.
Por\'em, caso necess\'ario, conv\'em cit\'a-lo(s) nesta parte e
n\~ao na se\c{c}\~ao seguinte chamada \textbf{Conclus\~oes}.

\section{The Last Question}
Nesta secção você deverá inserir o seu nome e apresentar a última questão:

\begin{itemize}
    \item nome :: inserir a última questão?
\end{itemize}


\section{Considerações finais}
\label{sec:consid}

Brevemente comentada no texto acima, nesta se\c{c}\~ao o
pesquisador (i.e. autor principal do trabalho cient\'ifico) deve
apresentar sua opini\~ao com respeito \`a pesquisa e suas
implica\c{c}\~oes. Descrever os impactos (i.e.
tecnol\'ogicos,sociais, econ\^omicos, culturais, ambientais,
políticos, etc.) que a pesquisa causa. N\~ao se recomenda citar
outros autores.

